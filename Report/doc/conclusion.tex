% !TEX root = ../Report.tex

In this project different approaches for lung segmentation were examined. The goal was to create a 3D segmentation of the lung from a CT image volume scan consisting of slices. The two deep convolutional neuronal network architectures DeepMedic and U-Net were configured and implemented to be trained on a set of clinical chest CT scans. The predictions of both models were evaluated using the Dice coefficient, the Hausdorff distance and the mean distance as metrics. Both models performed very well on the lung segmentation and created an accurate structure of the lungs and the bronchus. However, the largest difference between the two networks was the consistency of the segmentation of the more difficult to delineate bronchus which the DeepMedic model excelled at, however the training time for DeepMedic was a few hours longer.\newline
To put the work into a larger context it can be said that many differing architectures exist for the difficult and broad task of image segmentation in medical applications. One of the biggest challenges in image segmentation is getting appropriate and reliable data and preparing and augmenting this data for effective model generation for machine learning. On top of that, handling of the exceptionally large datasets and image volumes, reduction of training time and the allocation of computational resources make the task even more challenging and many approaches to these problems need to be assessed for each unique application.\newline
It can be concluded that machine learning today can create valuable new results in medical image processing. In the specific task of lung segmentation the next step after extracting the lung from the CT scans would be to mark cancerous nodules in the lung for cancer detection and treatment. This work is therefore a helpful first step to reach the goal of improved medical treatment through machine learning approaches.