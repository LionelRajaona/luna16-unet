% !TEX root = ../Report.tex

In this project different approaches for lung segmentation were examined. The goal was to create a 3D model of the lung from a stack of CT scan slices. The two convolutional neuronal network architectures DeepMedic and the U-Net were adjusted or implemented and trained on a set of CT scans. The predictions of both models were evaluated by incorporating the dice loss, the Hausdorff distance and the mean distance as metrics. Both models performed under all metrics outstandingly accurate and created a valuable structure of the lungs and the bronchus. An overall better model could not be found, each model had its advantages and disadvantages.\newline
To put the work in greater context it can be said that complex and well working architectures exist for the very specific task of image segmentation in medical applications. Therefore, one of the biggest challenges in segmentation are getting appropriate and reliable data and adjusting this data for machine learning. On top of that handling the big dataset, reduction of training time and the allocation of computational resources make the task even more challenging.\newline
It nevertheless can be concluded that machine learning today can create valuable results in medical image processing. In the specific task of lung segmentation the next step after extracting the lung from the CT scans would be the detection of ill nodules in the lung for cancer detection and treatment. This work is therefore a helpful first step to reach the goal of improved medical treatment through machine learning approaches.