% !TEX root = ../Report.tex

Semantic segmentation of anatomical structures from medical image volumes is a common subject of research in biomedical engineering. Traditional techniques predating deep learning for medical image segmentation include thresholding, atlas-based segmentation, and statistical shape modeling. A paper comparing the capabilities of thresholding, region-based, shape-based, anatomy-guided, and machine learning approaches for segmenting structures in the lungs has been done in 2015 \cite{comparison:article_typical}. This report found that while more computationally expensive than other approaches, machine learning based approaches have the most wide range of possible applications when used for segmentation of lung fields and lung diseases.
Another report finds that when using atlas based segmentation of the lungs with non-rigid registration on chest x-rays a Dice coefficient of up to 0.967 average can be reached [Lung Segmentation in Chest Radiographs Using Anatomical Atlases With Nonrigid Registration]. This Dice value is quite high but is found on much less detailed 2D x-ray slide segmentations.